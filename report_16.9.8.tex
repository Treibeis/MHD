% mnras_template.tex
%
% LaTeX template for creating an MNRAS paper
%
% v3.0 released 14 May 2015
% (version numbers match those of mnras.cls)
%
% Copyright (C) Royal Astronomical Society 2015
% Authors:
% Keith T. Smith (Royal Astronomical Society)

% Change log
%
% v3.0 May 2015
%    Renamed to match the new package name
%    Version number matches mnras.cls
%    A few minor tweaks to wording
% v1.0 September 2013
%    Beta testing only - never publicly released
%    First version: a simple (ish) template for creating an MNRAS paper

%%%%%%%%%%%%%%%%%%%%%%%%%%%%%%%%%%%%%%%%%%%%%%%%%%
% Basic setup. Most papers should leave these options alone.
\documentclass[fleqn,usenatbib]{mnras}

% MNRAS is set in Times font. If you don't have this installed (most LaTeX
% installations will be fine) or prefer the old Computer Modern fonts, comment
% out the following line
%\usepackage{newtxtext,newtxmath}
\usepackage{amsfonts}
% Depending on your LaTeX fonts installation, you might get better results with one of these:
%\usepackage{mathptmx}
%\usepackage{txfonts}

% Use vector fonts, so it zooms properly in on-screen viewing software
% Don't change these lines unless you know what you are doing
\usepackage[T1]{fontenc}
\usepackage{ae,aecompl}

%%%%% AUTHORS - PLACE YOUR OWN PACKAGES HERE %%%%%

%\usepackage{morefloats}% A pacakge which enables this text to contain more figures and tables.
\usepackage{grffile}% A pacakge which changes the algorithm to check for known extensions, so that we can insert pdf figures into this text.

% Only include extra packages if you really need them. Common packages are:
\usepackage{graphicx}	% Including figure files
\usepackage{amsmath}	% Advanced maths commands
\usepackage{amssymb}	% Extra maths symbols

%%%%%%%%%%%%%%%%%%%%%%%%%%%%%%%%%%%%%%%%%%%%%%%%%%

%%%%% AUTHORS - PLACE YOUR OWN COMMANDS HERE %%%%%

% Please keep new commands to a minimum, and use \newcommand not \def to avoid
% overwriting existing commands. Example:
%\newcommand{\pcm}{\,cm$^{-2}$}	% per cm-squared

%%%%%%%%%%%%%%%%%%%%%%%%%%%%%%%%%%%%%%%%%%%%%%%%%%

%%%%%%%%%%%%%%%%%%% TITLE PAGE %%%%%%%%%%%%%%%%%%%

% Title of the paper, and the short title which is used in the headers.
% Keep the title short and informative.
\title[Report on September 8]{The isothermal MHD shock conditions under (dynamical) isentropic/isothermal MHD energy conservation}

% The list of authors, and the short list which is used in the headers.
% If you need two or more lines of authors, add an extra line using \newauthor
\author[Bo-Yuan Liu]{Bo-Yuan Liu$^{1}$\thanks{E-mail: liu-by13@mails.tsinghua.edu.cn}
\\
% List of institutions
$^{1}$Department of Physics, Tsinghua University, Beijing, China(PRC)\\
}

% These dates will be filled out by the publisher
\date{Accepted XXX. Received YYY; in original form ZZZ}

% Enter the current year, for the copyright statements etc.
\pubyear{2016}

% Don't change these lines
\begin{document}
\label{firstpage}
\pagerange{\pageref{firstpage}--\pageref{lastpage}}
\maketitle
\begin{keywords}
MHD -- shock waves -- stars: formation -- supernovae: general -- ISM: clouds
\end{keywords}


%%%%%%%%%%%%%%%%%%%%%%%%%%%%%%%%%%%%%%%%%%%%%%%%%%

%%%%%%%%%%%%%%%%% BODY OF PAPER %%%%%%%%%%%%%%%%%%
 
\section{When the effective entropy is conserved}
This section shows how the equations for MHD shock conditions reduce from their general forms in a (sphericall symmetric) general polytropic magneto-fluid (under self-gravity) of \citet{wang2008dynamic} to those in an isothermal magneto-fluid of the present paper when $\gamma\rightarrow 1$.

The self-similar transformation in \citet{wang2008dynamic} (formulae (7)) is
\begin{gather}
\begin{split}
r&=k^{\frac{1}{2}}t^{n}x\ ,\quad u=k^{\frac{1}{2}}t^{n-1}v\ ,\quad \rho=\frac{\alpha}{4\pi Gt^{2}}\ ,\\
p&=\frac{kt^{2n-4}}{4\pi G}\beta\ ,\quad M=\frac{k^{\frac{3}{2}}t^{3n-2}m}{(3n-2)G}\ ,\quad \langle B^{2}_{t}\rangle=\frac{kt^{2n-4}}{G}\omega\ ,
\end{split}\label{e1}
\end{gather}
where the definitions of $u$, $\rho$, $p$ and $M$ are
the same with my work and the $B_{t}$ here is just the $B_{\parallel}$ in my draft. We further know from the MHD PDEs/ODEs that (equations (11) and (12) in \citet{wang2008dynamic}) 
\begin{gather}
\omega=h\alpha^{2}x^{2}\ ,\quad \beta=\alpha^{\gamma}m^{q}\ ,\label{e2}
\end{gather}
where $\gamma$ is the polytropic index and $q=2(n+\gamma-2)/(3n-2)$. Since for the polytropic MHD model in \citet{wang2008dynamic}, the effective entropy is conserved, which is only possible in our isothermal MHD model with $\gamma\rightarrow 1$. Therefore, as $\gamma\rightarrow 1$, $n\rightarrow 1$, $q\rightarrow 0$, $k=a^{2}$, and $h=\lambda$, the general polytropic MHD model just reduces to our isothermal MHD model, where $a$ is the isothermal speed of sound and $\lambda$ is the dimensionless parameter that characterises the strength of the magnetic field in our isothermal MHD model.

The shock conditions expressed by physical quantities are (expressions (17)-(20) in \citet{wang2008dynamic}\footnote{It here uses a pair of square brackets outside each expression enclosed to denote the difference between the upstream (marked by sub-script `u') and downstream (marked by subscript `d') quantities, as has been done conventionally for shock analyses.})
\begin{gather}
\left[\rho\left(u_{s}-u\right)\right]_{u}^{d}=0\ ,\label{e3}\\
\left[p+\rho\left(u_{s}-u\right)^{2}+\frac{\langle B_{t}^{2}\rangle}{8\pi}\right]_{u}^{d}=0\ ,\label{e4}\\
\left[\frac{\rho\left(u_{s}-u\right)^{3}}{2}+\frac{\gamma p\left(u_{s}-u\right)}{\gamma-1}+\frac{\langle B_{t}^{2}\rangle}{4\pi}\left(u_{s}-u\right)\right]_{u}^{d}=0\ ,\label{e5}\\
\left[\left(u_{s}-u\right)^{2}\langle B_{t}^{2}\rangle\right]_{u}^{d}=0\ ,\label{e6}
\end{gather}
of which euqation~(\ref{e5}) for (dynamical) MHD energy conservation\footnote{Actually, equation~(\ref{e5}) is based on the presupposition that gas around the shock layer experience an adiabatic process, i.e. $\partial s/\partial t+\bf{u}\cdot\nabla s$, in which there is no heat flux, and that is why we use the qualifier `dynamical'.} is not well formulated in the isothermal model, since the second term $\gamma p\left(u_{s}-u\right)/(\gamma-1)$ diverges as $\gamma\rightarrow 1$, and there is no relevant equation for MHD/HD energy conservation in the isothermal models of \citet{lou2014self}, \citet{yuLou2006} and my work. So what is the condition of MHD/HD energy conservation in the isothermal MHD/HD model? The following derivations will show that in the isothermal MHD model with $\gamma\rightarrow 1$, MHD energy is conserved (dynamically) only when both sides of the shock have the same temperature.


By formulae~(\ref{e1}) and (\ref{e2}), from equations~(\ref{e3})-(\ref{e6}), we obtain the shock conditions in self-similar variables (formulae~(64) in \citet{wang2008dynamic}):
\begin{gather}
\alpha_{1}\Gamma_{1}=\alpha_{2}\Gamma_{2}\ ,\label{e7}\\
\alpha_{1}^{2-n+\frac{3nq}{2}}x_{1}^{3q-2}\Gamma_{1}^{q}+\alpha_{1}\Gamma_{1}^{2}+\frac{h\alpha_{1}^{2}}{2}=left(\alpha_{2},x_{2},\Gamma_{2})\ ,\label{e8}\\
\Gamma_{1}^{2}+\frac{2\gamma}{\gamma-1}\alpha_{1}^{1-n+\frac{3nq}{2}}x_{1}^{3q-2}\Gamma_{1}^{q}+2h\alpha_{1}=left(\alpha_{2},x_{2},\Gamma_{2})\ ,\label{e9}
\end{gather}
where $left(\alpha_{2},x_{2},\Gamma_{2})$ denotes the formula on the left side of the equal sign with $\alpha_{1}$, $x_{1}$ and $\Gamma_{1}$ replaced by $\alpha_{2}$, $x_{2}$ and $\Gamma_{2}$, $\Gamma_{i}=n-v_{i}/x_{i}$, and $\tau=\sqrt{k_{2}/k_{1}}=x_{1}/x_{2}$, since $u_{s}=dr_{s}/dt=nk_{i}^{1/2}t^{n-1}x_{i}$ ($i=1,2$). And it is necessay to point out that equations~(\ref{e7})-(\ref{e9}) correspond to equations~(\ref{e3})-(\ref{e5}), respectively, while equation~(\ref{e6}) has been combined with equation~(\ref{e3}) to give $h_{1}=h_{2}=h$. We can further eliminate $\alpha_{2}$ in equations~(\ref{e8}) and (\ref{e9}) by equation~(\ref{e7}) to give (equations~(65) in \citet{wang2008dynamic})
\begin{gather}
\alpha_{1}^{2-n+\frac{3nq}{2}}x_{1}^{3q-2}\Gamma_{1}^{q}+\alpha_{1}\Gamma_{1}^{2}+\frac{h\alpha_{1}^{2}}{2}\notag\\
=\frac{\left(\alpha_{1}\Gamma_{1}\right)^{2-n+\frac{3nq}{2}}}{\Gamma_{2}^{2-n+\frac{(3n-2)q}{2}}}x_{2}^{3q-2}+\alpha_{1}\Gamma_{1}\Gamma_{2}+\frac{h\alpha_{1}^{2}\Gamma_{1}^{2}}{2\Gamma_{2}^{2}}\ ,\label{e10}\\
\frac{2\gamma}{(\gamma-1)}\alpha_{1}^{1-n+\frac{3nq}{2}}x_{1}^{3q-2}\Gamma_{1}^{q}+\Gamma_{1}^{2}+2h\alpha_{1}\notag\\
=\frac{2\gamma}{(\gamma-1)}\frac{\left(\alpha_{1}\Gamma_{1}\right)^{1-n+\frac{3nq}{2}}}{\Gamma_{2}^{1-n+\frac{(3n-2)q}{2}}}x_{2}^{3q-2}+\Gamma_{2}^{2}+2h\frac{\alpha_{1}\Gamma_{1}}{\Gamma_{2}}\ ,\label{e11}
\end{gather}
of which equation~(\ref{e11}) can be regarded to embody (dynamical) MHD energy conservation since it is directly related to equation~(\ref{e5}) which, as mentioned above, cannot be included in the $\gamma\rightarrow 1$ isothermal model in a straight forward manner, while equation~(\ref{e6}) corresponds to momentum conservation.

Now, recall the cubic equation (40) and the quadric equation (34) (whose real solutions correspond to possible shocks) of my draft for two-temperature and one-temperature isothermal MHD shock conditions
\begin{gather}
X^{3}+\tau^{2}X^{2}\beta_{1}-\left(1+\beta_{1}+\beta_{1}M_{1}^{2}\right)X+M_{1}^{2}\beta_{1}=0\ ,\label{e12}\\
X^{2}+\left(\beta_{1}+1\right)X-\beta_{1}M_{1}^{2}=0\ ,\label{e13}
\end{gather}
where 
\begin{gather}
\begin{split}
&\tau=a_{j}/a_{i}=x_{si}/x_{sj}\ ,
\ M_{1}=v_{i}-x_{si}\ ,
\ \beta_{1}=2/(\lambda x^{2}_{si}\alpha_{i})\ ,\\
&X=\alpha_{j}/\alpha_{i}=(v_{i}-x_{si})/[\tau (v_{j}-x_{sj})]\ .
%\\
%&M_{1}=v_{i}-x_{si}\ ,
%\qquad \beta_{1}=2/(\lambda x^{2}_{si}\alpha_{i})\ .
\end{split}\label{e14}
\end{gather}
Our goal is to find the relations among equations~(\ref{e10})-(\ref{e11}) and equations~(\ref{e12})-(\ref{e13}). For the relations among the variables of these two groups of equations, when we set $(i,j)=(1,2)$\footnote{To be consistent, we also have $x_{s1}=x_{1}$ and $x_{s2}=x_{2}$.} in formulae~(\ref{e14}), as $\gamma\rightarrow 1$, $n\rightarrow 1$, $q\rightarrow 0$, $k=a^{2}$, and $h=\lambda$, it turns out that
\begin{gather}
\beta_{1}=\frac{2}{hx_{1}^{2}\alpha_{1}}\ ,\quad M_{1}=-x_{1}\Gamma_{1}\ ,\quad X=\frac{\Gamma_{1}}{\Gamma_{2}}\ .\label{e15}
\end{gather}
Then through simple algebra and equations~(\ref{e15}), the reduced (isothermal) version of equation~(\ref{e10})
\begin{gather}
\alpha_{1}x_{1}^{-2}+\alpha_{1}\Gamma_{1}^{2}+\frac{h\alpha_{1}^{2}}{2}\notag\\
-\frac{\alpha_{1}\Gamma_{1}}{\Gamma_{2}}\left(\frac{x_{1}}{\tau}\right)^{-2}-\alpha_{1}\Gamma_{1}\Gamma_{2}-\frac{h\alpha_{1}^{2}\Gamma_{1}^{2}}{2\Gamma_{2}^{2}}=0 \label{e16}
\end{gather}
can be easily put into the form
\begin{gather}
\alpha_{1}x_{1}^{-2}\lbrace 1+\left(x_{1}\Gamma_{1}\right)^{2}+\frac{h\alpha_{1}x_{1}^{2}}{2}\notag\\
-\tau^{2}\frac{\Gamma_{1}}{\Gamma_{2}}-\left(x_{1}\Gamma_{1}\right)^{2}\frac{\Gamma_{2}}{\Gamma_{1}}-\frac{h\alpha_{1}x_{1}^{2}}{2}\left(\frac{\Gamma_{1}}{\Gamma_{2}}\right)^{2}\rbrace=0\ ,i.e.\notag\\
1+M_{1}^{2}+\frac{1}{\beta_{1}}-\tau^{2}X-\frac{M_{1}^{2}}{X}-\frac{1}{\beta_{1}}X^{2}=0\ ,\label{e17}
\end{gather}
which is exactly equation~(\ref{e12}) for two-temperature isothermal MHD shock conditions divided by $-X\beta_{1}$. \textbf{Here we do not consider equation~(\ref{e11}) for (dynamical) MHD energy conservation at all. So the conclusion is that for two-temperature isothermal MHD shocks, the MHD energy does not conserve.} Then what about one-temperature MHD shocks?

Actually, we can eliminate $x_{2}$ in equations~(\ref{e10}) and (\ref{e11}) to give (equations~(66) and (67) in \citet{wang2008dynamic})
\begin{gather}
\lbrace \frac{(\gamma+1)}{2\gamma}\Gamma_{2}^{2}-\left(\alpha_{1}^{1-n+\frac{3nq}{2}}x_{1}^{3q-2}\Gamma_{1}^{q-1}+\frac{\gamma-1}{2\gamma}\Gamma_{1}+\frac{h\alpha_{1}}{2\Gamma_{1}}\right)\Gamma_{2}\notag\\
-\frac{2-\gamma}{2\gamma}h\alpha_{1}\rbrace \left(\Gamma_{2}-\Gamma_{1}\right)=0\ ,\label{e18}
\end{gather}
which can be reduced to the isothermal version
\begin{gather}
\left[\Gamma_{2}^{2}-\left(x_{1}^{-2}\Gamma_{1}^{-1}+\frac{h\alpha_{1}}{2\Gamma_{1}}\right)\Gamma_{2}-\frac{h\alpha_{1}}{2}\right]\left(\Gamma_{2}-\Gamma_{1}\right)=0\ .\label{e19}
\end{gather}
By equations~(\ref{e15}), we can write this equation~(\ref{e19}) as
\begin{gather}
\left[\left(\Gamma_{1}x_{1}\right)^{2}\left(\frac{\Gamma_{2}}{\Gamma_{1}}\right)^{2}-\left(1+\frac{h\alpha_{1}x_{1}^{2}}{2}\right)\frac{\Gamma_{2}}{\Gamma_{1}}-\frac{h\alpha_{1}x_{1}^{2}}{2}\right]\notag\\
\times \left(1-\frac{\Gamma_{1}}{\Gamma_{2}}\right)\Gamma_{2}x_{1}^{-2}=0\ , i.e.\notag\\
\left[M_{1}^{2}\frac{1}{X^{2}}-\left(1+\frac{1}{\beta_{1}}\right)\frac{1}{X}-\frac{1}{\beta_{1}}\right](1-X)=0\ ,\label{e20}
\end{gather}
which is exactly equation~(\ref{e13}) multiplied by a factor $(X-1)/\left(X^{2}\beta_{1}\right)$, and we can throw away the factor $\left(1-X\right)$, since $X=1$ gives trivial results. This outcome indicates that \textbf{if we take into account (dynamical) MHD energy conservation by combining equations~(\ref{e10}) and (\ref{e11}) to attain equation~(\ref{e18}), the shock condition in the polytropic model will reduce to the one-temperature shock condition in the isothermal model under $\gamma\rightarrow 1$}. We may further conclude that equation~(\ref{e5}) for (dynamical) MHD energy conservation in the polytropic model is consistent with the very restriction that $a_{1}=a_{2}$ (i.e. $\left(k_{B}T_{u}/\mu\right)^{1/2}=a_{u}=a_{d}=\left(k_{B}T_{d}/\mu\right)^{1/2}$, $T_{u}=T_{d}$: both sides of the shock have the same temperature) in the $\gamma\rightarrow 1$ isothermal model. 

\section{When the effective entropy is not conserved: a `new' type of isothermal MHD shocks}
\label{s2}
The problem of non-conservation of (dynamical)\footnote{The qualifier `dynamical' here denotes that there is no heat flux.} MHD energy comes from the condition that $\gamma\rightarrow 1$ for an isothermal gas. So why shall $\gamma\rightarrow 1$ for an isothermal gas, if we regard it as the ratio of the isobaric heat capacity and the isovolumetric heat capacity which can be larger than 1 (e.g. $4/3$ for the relativistic perfect gas and $5/3$ for the non-relativistic perfect gas)? The answer is to conserve the effective entropy. 

In the polytropic model mentioned above, the equation of effective entropy conservation (equation~(6) in \citet{wang2008dynamic}) is
\begin{gather}
\left(\frac{\partial}{\partial t}+u\frac{\partial}{\partial r}\right)\left(\mathrm{ln}\frac{p}{\rho^{\gamma}}\right)=0\ ,\label{e21}
\end{gather}
from which and other MHD ODEs, we can obtain the relation $\beta=\alpha^{\gamma}m^{q}$ (see equations~(\ref{e2})). If meanwhile the gas is isothermal, i.e. $p=a^{2}\rho$, the self-similar transformation~(\ref{e1}) reduces to that (expressions~(6) in my work)\footnote{$r=atx$.}
\begin{gather}
\begin{split}
&\rho(r,t)=\frac{\alpha(x)}{4\pi Gt^{2}},
 \qquad\quad M(r,t)=\frac{a^{3}t}{G}m(x)\ ,\\
&u(r,t)=av(x), \qquad\quad B_{||}(r,\ t)=\frac{ab(x)}{\sqrt{G}t}\ .
\end{split}\label{e22}
\end{gather}
Therefore, we must have $n=1$, $k=A^{2}=\gamma a^{2}$ (, where $A$ is the adiabatic sound speed), $h=\lambda$, $\omega=b$, and $\alpha^{\gamma}m^{q}=\alpha/\gamma$, which means that $\gamma\rightarrow 1$ and $q=0$, then we arrive at the case of the last section that the effective entropy is conserved.

Actually, in my work as well as \citet{lou2014self}, \citet{yuLou2005} and \citet{yuLou2006}, equation~(\ref{e21}) does not show up, but the effective entropy is still conserved when $\gamma\rightarrow 1$ \textbf{for the model itself}, since $s=\mathrm{ln}\left(p/\rho^{\gamma}\right)=2\mathrm{ln}a=const.$. But for two-temperature shocks, because $a_{u}\neq a_{d}$, the effective entropy varies across the shock front. 
%The conclusion here is that \textbf{those two-temperature MHD/HD isothermal shocks studied by our present work as well as \citet{lou2014self}, \citet{yuLou2005} and \citet{yuLou2006} broke the conservation of both MHD/HD energy and the effective entropy}.

If we discard the conservation of the effective entropy $\mathrm{ln}\left(p/\rho^{\gamma}\right)$ \textbf{for the model itself} and regard $\gamma>1$ as an adjustable parameter of the model, equation~(\ref{e5}) is no longer badly formulated, and we can include it into our model to derive the isothermal MHD shock condition that conserves MHD energy (dynamically), but this leads to a `new' type of isothermal shocks, which is beyond the range of our present paper.
%\footnote{It seems that I am in a hurry  to get the present paper submitted and accepted, so that it can help me apply for phD positions in the USA. Maybe this is true... Anyway, even though I am interested at this `new' type of isothermal MHD shocks, it will take a lot of time to study it throughly. I believe that they will produce physical results with the good feature that MHD energy conservation is considered. It may become an individual topic for another paper, unless someone else has already studied relevant problems.}.
By the self-similar transformation~(\ref{e22}), we can wirte equation~(\ref{e5}) for (dynamical) MHD energy conservation in self-similar variables
\begin{gather}
a_{i}^{2}\left[\frac{\left(v_{i}-x_{si}\right)^{2}}{2}+\frac{\gamma}{\gamma-1}+\lambda\alpha_{i}x_{si}^{2}\right]\notag\\
\qquad =a_{j}^{2}\left[\frac{\left(v_{j}-x_{sj}\right)^{2}}{2}+\frac{\gamma}{\gamma-1}+\lambda\alpha_{j}x_{sj}^{2}\right]\ ,\label{e23}
\end{gather}
where all the variables involved follow the same definitions of my draft and we have used equation~(\ref{e3}) to simplify the formula. This equation is shown by formulae~(\ref{e14}) to be equivalent to
\begin{gather}
X^{3}+\left[\frac{\left(\tau^{2}-1\right)\gamma}{2(\gamma-1)}\beta_{1}-1-\frac{M_{1}^{2}\beta_{1}}{4}\right]X^{2}+\frac{M_{1}^{2}\beta_{1}}{4}=0\ ,\label{e24}
\end{gather}
which can be combined with equation~(\ref{e12}) for momentum conservation to eliminate $\tau$ and give an equation for the complete isothermal shock condition with (dynamical) MHD energy conservation, that is independent of the strength of the shock $\tau$, which is similar to what has been done in \citet{wang2008dynamic} for the polytropic model. Finally we obtain a quadratic equation
\begin{gather}
f(X)=\frac{2-\gamma}{(\gamma-1)\beta_{1}}X^{2}+\left[\frac{M_{1}^{2}}{2}+\frac{\gamma}{\gamma-1}\left(1+\frac{1}{\beta_{1}}\right)\right]X\notag\\
\qquad -\frac{(\gamma+1)M_{1}^{2}}{2(\gamma-1)}=0\ ,\label{e25}
\end{gather}
for which we have thrown away the trivial factor $X-1$. When $1<\gamma<2$, the coefficient of the $X^{2}$ term  $(2-\gamma)/\left[(\gamma-1)\beta_{1}\right]$ is positive, and we have
\begin{gather}
f(X=0)=-\frac{(\gamma+1)M_{1}^{2}}{2(\gamma-1)}<0\ ,\label{e26}\\
f(X=1)=\frac{1}{\gamma-1}\left(\gamma+\frac{2}{\beta_{1}}-M_{1}^{2}\right)\ .\label{e27}
\end{gather}
Therefore, quadratic equation~(\ref{e25}) always has one positive root $X_{+}$ and one negative root $X_{-}$, and only the positive one is physical. When the upstream is given, i.e. $(i,\ j)=(u,\ d)$, we require that $X_{+}>1$, which means that $|u_{u}-u_{s}|>|u_{d}-u_{s}|$, where $|u_{u}-u_{s}|$ and $|u_{d}-u_{s}|$ are the upstream flow speed and the downstream flow speed relative to the shock front \citep{FM}, so we must have 
\begin{gather}
f(X=1)>0\ ,i.e.\quad \gamma+\frac{2}{\beta_{1}}-M_{1}^{2}<0\ ,\label{e28}
\end{gather}
which exactly corresponds to the physical requirement that the upstream flow speed relative to the shock speed must exceed the upstream fast magneto-sonic speed, i.e. $a_{u}^{2}\left(v_{u}-x_{su}\right)^{2}>A_{u}^{2}+v_{Au}^{2}$, where $v_{Au}=\sqrt{\lambda\alpha_{u}}\cdot x_{su}a_{u}$, and $A_{u}=a_{u}\gamma^{1/2}$ is the upstream adiabatic sound speed. Once attain the positive root $X=X_{+}>1$, we can calculate $\tau$ (which is no longer an adjustable parameter now) by
\begin{gather}
\tau^{2}=\frac{X-1}{X}\left(\frac{M_{1}^{2}}{X}-\frac{1+X}{\beta_{1}}\right)+\frac{1}{X}\ ,\label{e29}
\end{gather}
the right side of which must be larger than 1 due to the second law of thermodynamics, given the upstream.

When the properties of the downstream are given, i.e. $(i,\ j)=(d,\ u)$, we require that $X=X_{+}<1$, and $\tau<1$, thus, $f(X=1)>0$, and the right hand side of formula~(\ref{e29}) should be positive and smaller than 1. We can further prove that $a_{u}^{2}\left(v_{u}-x_{su}\right)^{2}>A_{u}^{2}+v_{Au}^{2}$ is still valid, which is now equivalent to $\gamma+2/\beta_{2}<M_{2}^{2}$, where the definitions of $M_{2}$ and $\beta_{2}$ are similar to those of $M_{1}$ and $\beta_{1}$, i.e. $M_{2}\equiv u_{2}/a_{2}=M_{1}/(X\tau)$ and $\beta_{2}\equiv 8\pi p_{2}/\left(B_{2}^{2}\right)=\beta_{1}\tau^{2}/X$. By these definitions and the expression of $\tau^{2}$~(\ref{e29}), we have
\begin{gather}
\left(\gamma+\frac{2}{\beta_{2}}-M_{2}^{2}\right)\tau^{2}=\tau^{2}\gamma+\frac{2X}{\beta_{1}}-\frac{M_{1}^{2}}{X^{2}}\notag\\
\qquad =\left[\frac{\gamma\left(X-1\right)-1}{X^{2}}\right]M_{1}^{2}+\frac{1}{\beta_{1}}\left[(2-\gamma)X+\frac{\gamma}{X}\right]+\frac{\gamma}{X}\ .\label{e30}
\end{gather}
Since $X<1$, and $1<\gamma<2$, from equation~(\ref{e25}), we note down an inequality
\begin{gather}
\frac{(\gamma+1)M_{1}^2}{2X^{2}}-\frac{(\gamma-1)M_{1}^{2}}{2X}-\frac{\gamma}{X}\notag\\
\qquad =\left[(2-\gamma)+\frac{\gamma}{X}\right]\frac{1}{\beta_{1}}> \left[(2-\gamma)X+\frac{\gamma}{X}\right]\frac{1}{\beta_{1}}\ ,\label{e31}
\end{gather}
which is substituted to formula~(\ref{e30}) to eliminate the second term of the right side and gives
\begin{gather}
\left(\gamma+\frac{2}{\beta_{2}}-M_{2}^{2}\right)\tau^{2}<\frac{(\gamma+1)(X-1)M_{1}^{2}}{2X^{2}}<0\ ,i.e.\notag\\
\qquad \gamma+\frac{2}{\beta_{2}}-M_{2}^{2}<0\ .\label{e32}
\end{gather}
The above analysis implies that equation~(\ref{e25}) is the correct equation for the isothermal MHD shock condition with (dynamical) MHD energy conservation, from which we obtained the prefered `one to one' relation between the upstream and the downstream.

Besides, if we have $\gamma\rightarrow 1$, equation~(\ref{e25}) can be put into the form
\begin{gather}
(\gamma-1)f(X)\beta_{1}\notag\\
\qquad =(2-\gamma)X^{2}+\left[\frac{(\gamma-1)M_{1}^{2}\beta_{1}}{2}+\gamma\left(1+\beta_{1}\right)\right]X\notag\\
\qquad -\frac{(\gamma+1)\beta_{1}M_{1}^{2}}{2}=0\ ,i.e.\notag\\
\qquad X^{2}+\left(1+\beta_{1}\right)X-\beta_{1}M_{1}^{2}=0\ ,\label{e33}
\end{gather}
which is exactly the quadratic equation~(\ref{e13}) for the one-temperature MHD shock condition. By this equation and equation~(\ref{e12}), we can further show that
\begin{gather}
\frac{X^{2}-1}{\beta_{1}}+X\tau^{2}-1+\frac{1-X}{X}M_{1}^{2}\notag\\
\qquad = \frac{X^{2}-1}{\beta_{1}}+X\tau^{2}-1+(1-X)+\frac{1-X^{2}}{\beta_{1}}\notag\\
\qquad =X\left(\tau^{2}-1\right)=0\ ,i.e.\notag\\
\qquad\tau=1\ ,\label{e35}
\end{gather}
where the first line is just the left hand side of equation~(\ref{e13}) divided by $\beta_{1}X$, and for the second line we have used equation~(\ref{e33}) multiplied by $(1/X-1)/\beta_{1}$. This outcome supports our argument that (dynamical) MHD energy conservation for isotherm MHD shocks under $\gamma\rightarrow 1$ is consistent with the very restriction that $a_{1}=a_{2}$ (i.e. $\left(k_{B}T_{u}/\mu\right)^{1/2}=a_{u}=a_{d}=\left(k_{B}T_{d}/\mu\right)^{1/2}$, $T_{u}=T_{d}$: both sides of the shock have the same temperature).

I have already written and tested the codes for this new type of isothermal MHD shocks under (dynamical) MHD energy conservation. By now, I have verified that the shock condition~(\ref{e25}) is indeed reversible, but it would be more difficult to construct relevant shock solutions, in that $\tau$ is not adjustable here.

\section{(dynamical) Isothermal MHD energy conservation for isothermal MHD shocks}
Since equation~(\ref{e23}) for MHD energy conservation is based on the presupposition that gas around the shock layer experience an adiabatic process (of no `entropy source'), i.e. $\partial S/\partial t+{\bf{u}} \cdot\nabla S=\partial S/\partial t+u\partial S/\partial r=0$, the MHD energy conservation considered in the last section can be called (dynamical) isentropic MHD energy conservation. 

In this section, I investigate another version of MHD energy conservation, i.e. (dynamical) isothermal MHD energy conservation, which assumes that gas near the shock layer shares the same temperature (though the temperatures of different sides of the shock can be different). According to the energy flux formula (i.e. equations~(23) of \citet{lou2004envelope}), the isothermal counterpart of equation~(\ref{e5}) is
\begin{gather}
\left[\frac{\left(u_{s}-u\right)^{2}}{2}+a^{2}\mathrm{ln}\left(\frac{\rho}{\rho_{c}}\right)+\frac{\langle B_{t}^{2}\rangle}{4\pi \rho}\right]_{u}^{d}=0\ ,\label{e35}
\end{gather} 
in which we have used the expression of entropy per unit mass $S$, i.e.
\begin{gather}
S=S_{0}+\frac{R}{\gamma+1}\mathrm{ln}\left(\frac{a^{2}\rho^{1-\gamma}}{\gamma-1}\right)=\frac{R\gamma}{\gamma-1}-R\mathrm{ln}\left(\frac{\rho}{\rho_{c}}\right)\ ,\label{e36}
\end{gather}
where $RT=a^{2}$. The problem here is that we have to do self-similar transformation of $\rho_{c}$ to express equation~(\ref{e35}) in self-similar variables, while choice of such transformation is sort of arbitrary. And I always doubt the significance of studying such MHD energy conservation in the isothermal model. Actually, we have already assumed that different parts (which are not separated by shocks) of the system share the same invariant temperature, which implies that heat conduction may be efficient in our model. Thus, considering other possibly effective mechanisms involved around the shock (e.g. radiation transfer, cosmic rays, and chemical reaction), it should be allowed that non-zero heat flux exists across the shock front, so that MHD energies (per unit mass) of the two sides of the shock can be different. 

Anyway, I just try two possible types of self-similar transformation: (1) $\rho_{c}^{-1}=4\pi Gt^{4-2\gamma}r^{2\gamma-2}a^{2\gamma-2}$, (2) $\rho_{c}^{-1}=4\pi Gt^{2}$, where the first ensures that $S_{0}$ is independent of $a$, while the second means that $\rho_{c}$ itself is independent of $a$. Then I can put equation~(\ref{e35}) into the conventional form of our paper by formulae~(\ref{e14}):
\begin{gather}
\frac{2(X-1)}{\beta_{1}}+\frac{M_{1}^{2}}{2}\left(\frac{1}{X^{2}}-1\right)+\tau^{2}\mathrm{ln}X+\tau^{2}(1-\gamma)\mathrm{ln}\tau^{2}\notag\\
\qquad +\left(\tau^{2}-1\right)\left[\mathrm{ln}\alpha_{1}+(2\gamma-2)\mathrm{ln}x_{1}\right]=0\ ,\label{e37}\\
\frac{2(X-1)}{\beta_{1}}+\frac{M_{1}^{2}}{2}\left(\frac{1}{X^{2}}-1\right)+\tau^{2}\mathrm{ln}X+\left(\tau^{2}-1\right)\mathrm{ln}\alpha_{1}=0\ ,\label{e38}
\end{gather}
of which equation~(\ref{e37}) corresponds to $\rho_{c}^{-1}=4\pi Gt^{4-2\gamma}r^{2\gamma-2}a^{2\gamma-2}$ (Condition 1) , while equation~(\ref{e38}) $\rho_{c}^{-1}=4\pi Gt^{2}$ (Condition 2). These equations are transcendental and cannot be evaluated analytically. I have to combine them with equation~(\ref{e12}) for momentum conservation to solve $X$ and $\tau$ numerically. 

Note that after throwing away the first term with $\beta_{1}$, equation~(\ref{e38}) (Condition 2) is equivalent to your formula (without magnetic fields) for some conservation that 
\begin{gather}
\frac{u_{1}^{2}}{2}+a_{1}^{2}\mathrm{ln}(\alpha_{1})=\frac{u_{2}^{2}}{2}+a_{2}^{2}\mathrm{ln}(\alpha_{2})\ .\label{e39}
\end{gather}
Here the shock condition is indeed independent of $\gamma$. I conducted numerical calculations of Condition 2 shock solutions for $\lambda=10^{-10}\approx 0$ by adding shocks at different $x_{su}$s to Type 1 eigensolutions starting from some $x_{*}(u)$s on the SCL, which is accurate enough to resemble the pure hydrodynamic case\footnote{I have verified this by comparing my results under $\lambda=10^{-10}$ with those of \citet{lou2014self} without magnetic fields. }. I found that it is actually possible to find `reasonable' solutions with both $\tau>1$ and $X>1$, given the upstream properties. Detailed results are shown in Figure~\ref{1}-\ref{6}. In these figures, some solutions hit the SCL and show instability, which is not our main concern here, since we only want to know features of the shock and are not at that stage of constructing complete shock solutions. So please ignore those meaningless curves after hitting the SCL with instability.

Besides, I found something strange in \citet{lou2014self}, i.e. its equation~(20) for the change of specific entropies 
\begin{gather}
s_{d}-s_{u}=c_{v}\mathrm{ln}\left(\frac{P_{d}}{\rho_{d}}\right)-c_{v}\mathrm{ln}\left(\frac{P_{u}}{\rho_{u}}\right)=-2c_{v}\mathrm{ln}\mathcal{J}\ ,\label{e40}
\end{gather}
where $P_{u}$ and $P_{d}$ are pressures of the upstream side and the downstream side, $c_{v}$ is the specific heat at constant volume as a global constant, and $\mathcal{J}=a_{u}/a_{d}$. Here $s_{d}>s_{u}$ leads to $\mathcal{J}<1$, i.e. $a_{d}>a_{u}$. I have never seen such a expression for the so-called specific entropy. I just know that the effective entropy of the gas should be $s=\mathrm{ln}(p/\rho^{\gamma})$ which conflicts with the above equation. Actually, I discussed with Yu-Kai and concluded that since we are considering an isothermal dynamical model, there must be some external heat reservoir to keep the temperature of the system constant, i.e. the interchange of heat as well as entropy between the system and the reservoir always exists. As a result, the requirement that $s_{d}-s_{u}>0$ is baseless as it does not consider entropy change of the reservoir. However, the requirement that $a_{d}>a_{u}$ remains valid, though it is not based on the second law of thermodynamics, and we do not need to correlate it with the change of entropy across the shock front (as equation~(\ref{e40}) does, which is problematic). We may say that isothermal MHD/HD shocks are triggered by fluctuations of the environment which suddenly heat and compress the gas in certain region, and this region is exactly the initial downstream side on whose boundary the shock emerges and propagates toward the upstream side with relatively lower temperature and density\footnote{It is not impossible to imagine the opposite process in which environment fluctuations cool and dilute the gas. But I think that such a case is unlikely to occur in the context of astrophysics that we are concerned with (e.g. star formation and supernovae).}.

If we still require that $s_{d}-s_{u}>0$ (i.e. $S_{d}>S_{u}$), and use the correct expression~(\ref{e36}) for entropy per unit mass $S$, I can show that Condition 2 is itself contradictory, since here the requirement of $S_{d}-S_{u}>0$ becomes $R\mathrm{ln}(\alpha_{u}/\alpha_{d})>0$. And this contradicts the natural requirement for shock propagation that $|u_{s}-u_{u}|>|u_{s}-u_{d}|$ (which is equivalent to $\alpha_{u}/\alpha_{d}<1$). This outcome implies that it is to ask for trouble to talk about the entropy for an isothermal system. 

%Unfortunately, by limited numerical trials (for $\gamma=5/3$, and $\lambda=0.1$), for both Condition 1 and Condition 2, I cannot find any isothermal MHD shock under such (dynamical) isothermal MHD energy conservation. The problem is i) in  many regions of the solution space the transcendental equation set have no solution al all, and ii) when we have $X>1$, we always meanwhile have $\tau<1$ (and vice via: $X<1$ with $\tau>1$), which violates the second law of thermodynamics\footnote{Later on I realized that Condition 2 is itself contradictory, since here the requirement of $S_{d}-S_{u}>0$ becomes $\mathrm{ln}(\alpha_{u}/\alpha_{d})>0$. And this contradicts the natural requirement for shock propagation that $|u_{s}-u_{u}|>|u_{s}-u_{d}|$ (which is equivalent to $\alpha_{u}/\alpha_{d}<1$). }. Figure~\ref{1}-\ref{4} show some \textbf{FALSE} isothermal MHD shock solutions under such (dynamical) isothermal MHD energy conservation. (Even though these \textbf{FALSE} solutions are unphysical, they are indeed \textbf{reversible}.) I also plot some solutions under (dynamical) isentropic MHD energy conservation prescribed by the last section in Figure~\ref{5}-\ref{e6}, which show that both Type 1 and Type 2 two temperature shock condition can produce some results that do not break (dynamical) isentropic MHD energy conservation. In these figures, some solutions hit the MSCL and show instability, which is not our main concern here, since we only want to know features of the shock and are not at that stage of constructing complete shock solutions. So please ignore those meaningless curves after hitting the MSCL with instability.

\begin{figure}
\centering
\includegraphics[width=1.0\columnwidth]{shock7.pdf}
\caption{$-v$ versus $x$ plots of some Shocks under Condition 2 (dynamical) isothermal MHD energy conservation for $\lambda=10^{-10}\approx 0$ and $x_{*}(u)=0.01$ (see Figure~\ref{7} for relevant $\mathcal{P}$ versus $\mathcal{V}$ diagram). Here we have both $\tau>1$ and $X>1$. It is found that only when $x_{su}$ is not too far away from $x_{*}(u)$, the transcendental equation set of Condition 2 has solutions.}
\label{1}
\end{figure}

\begin{figure}
\centering
\includegraphics[width=1.0\columnwidth]{shock8.pdf}
\caption{(To be compared with Figure~\ref{10}) $-v$ versus $x$ plots of some Shocks under Condition 2 (dynamical) isothermal MHD energy conservation for $\lambda=10^{-10}\approx 0$ and $x_{*}(u)=0.1$ (see Figure~\ref{7} and \ref{8} for relevant $\mathcal{P}$ versus $\mathcal{V}$ diagrams). Here we also have both $\tau>1$ and $X>1$. Interestingly, we found a Shock ($x_{su}=0.22$) whose downstream is in the supersonic region, and thus, it corresponds to Type 1 two-temperature shock condition ($\mathcal{V}_{*}(2)$ in Figure~\ref{8}). Now we know that both Type 1 and Type 2 two-temperature shock condition can produce some results that obey the conservation law considered here. The vacancies in the sequence of $x_{su}$s mean that no solution of the transcendental equation set is found. Again, if $x_{su}-x_{*}(u)$ is too large, the transcendental equation set of Condition 2 has no solution. }
\label{2}
\end{figure}

\begin{figure}
\centering
\includegraphics[width=1.0\columnwidth]{shock9.pdf}
\caption{(To be compared with Figure~\ref{11}) $-v$ versus $x$ plots of some Shocks under Condition 2 (dynamical) isothermal MHD energy conservation for $\lambda=10^{-10}\approx 0$ and $x_{*}(u)=0.4$ (see Figure~\ref{7} and \ref{8} for relevant $\mathcal{P}$ versus $\mathcal{V}$ diagrams). Here we also have both $\tau>1$ and $X>1$. Again, we can see a Shock ($x_{su}=0.52$) corresponding to Type 1 two-temperature shock condition ($\mathcal{V}_{*}(2)$ in Figure~\ref{8}). \textbf{If $x_{su}-x_{*}(u)$ is too large, the transcendental equation set of Condition 2 has no solution, which is shown by more results an universal feature}. }
\label{3}
\end{figure}

\begin{figure}
\centering
\includegraphics[width=1.0\columnwidth]{shock10.pdf}
\caption{(To be compared with Figure~\ref{12}) $-v$ versus $x$ plots of some `unreasonable' Shocks under Condition 2 (dynamical) isothermal MHD energy conservation for $\lambda=10^{-10}\approx 0$ and $x_{*}(u)=0.8$ ($\mathcal{V}_{*}(2)$ in the exemplar $\mathcal{P}$ versus $\mathcal{V}$ diagram Figure~\ref{9}). \textbf{Here $X>1$, however, $\tau<1$, which is also true for other $x_{*}(u)$s larger than 0.8}.}
\label{4}
\end{figure}

\begin{figure}
\centering
\includegraphics[width=1.0\columnwidth]{shock11.pdf}
\caption{(To be compared with Figure~\ref{13}) $-v$ versus $x$ plots of some `unreasonable' Shocks under Condition 2 (dynamical) isothermal MHD energy conservation for $\lambda=10^{-10}\approx 0$ and $x_{*}(u)=1.2$ ($\mathcal{V}_{*}(2)$ in the exemplar $\mathcal{P}$ versus $\mathcal{V}$ diagram Figure~\ref{9}). Here $X>1$, however, $\tau<1$. Please ignore those meaningless curves after hitting the SCL with instability.}
\label{5}
\end{figure}

\begin{figure}
\centering
\includegraphics[width=1.0\columnwidth]{shock12.pdf}
\caption{(To be compared with Figure~\ref{14}) $-v$ versus $x$ plots of some `unreasonable' Shocks under Condition 2 (dynamical) isothermal MHD energy conservation for $\lambda=10^{-10}\approx 0$ and $x_{*}(u)=2.0$ ($\mathcal{V}_{*}(2)$ in the exemplar $\mathcal{P}$ versus $\mathcal{V}$ diagram Figure~\ref{9}). Here $X>1$, however, $\tau<1$. Please ignore those meaningless curves after hitting the SCL with instability.}
\label{6}
\end{figure}

\section{Interpretation of the numerical results}
In this section, I apply the analysis of Yu-Kai to our self-similar solutions and interpret the features found in the above numerical results as well as some new results presented below. The evaluation is narrowed down to the case in which properties of the upstream is given, that is to say, $u_{1}/a_{1}=(x_{su}-v_{u})>1$ (i.e. the upstream side is in the supersonic region of the $-v$ versus $x$ diagram). To arrive at such a (sort of complicated) interpretation, we need some preparatory work:

I will use the $\mathcal{P}$ versus $\mathcal{V}$ diagram a lot, so the first step is to write $\mathcal{P}$ and $\mathcal{V}$ in self-similar parameters $\tau$ and $X$. In our self-similar model, if $(i,\ j)=(1,\ 2)$, from formulae~(\ref{e14}), we have 
\begin{gather}
\tau=\frac{a_{2}}{a_{1}}\ ,\qquad X=\frac{\alpha_{2}}{\alpha_{1}}=\frac{\rho_{2}}{\rho_{1}}\ ,\label{e41}
\end{gather}
while following Yu-Kai's definitions,
\begin{align}
\mathcal{V}=\frac{\rho_{1}}{\rho_{2}}=\frac{u_{2}}{u_{1}}=\frac{1}{X}\ ,\qquad\mathcal{P}=\frac{p_{2}}{p_{1}}=\frac{a_{2}^{2}}{a_{1}^{2}}\frac{\rho_{2}}{\rho_{1}}=\frac{\tau^{2}}{\mathcal{V}}\ .\label{e42}
\end{align}
Finally, $\tau^{2}=\mathcal{PV}$, and $X=1/\mathcal{V}$.

The second step is to know how the $\mathcal{P}$ versus $\mathcal{V}$ diagram reflects the qualitative features of our self-similar solutions. Given $\rho_{1}$ and $p_{1}$, we obtain $\rho_{1}$ and $p_{2}$ from certain point $(\mathcal{V}_{*},\ \mathcal{P}_{*})$ in the $\mathcal{P}$ versus $\mathcal{V}$ diagram (see Figure~\ref{7}-\ref{9}). This point is the intersection point of the blue dashed straight line named `conservation' as 
\begin{align}
-\frac{\mathcal{P}-1}{\mathcal{V}-1}=R_{1}=\frac{u_{1}^{2}}{a_{1}^{2}}\label{e43}
\end{align}
(denoting mass and momentum conservation) and the blue solid curve `shock isothermal' 
\begin{align}
\mathcal{P}=\frac{(\mathcal{V}+1)/2+\xi_{1}}{(\mathcal{V}+1)/2+\mathcal{V}\mathrm{ln}\mathcal{V}+\xi_{1}\mathcal{V}}=F_{\xi_{1}}(\mathcal{V})\ ,\label{e44}
\end{align}
where $R_{1}$ should be larger than 1 if properties of the upstream is known (the dotted curve `SCL' corresponds to $R_{1}=1$), and $\xi_{1}=\mathrm{ln}(\rho_{0}/\rho_{1})=-\mathrm{ln}\alpha_{1}$ for Condition 2 prescribed by equation~(\ref{e39}) as a conservation law. The values of $\xi_{1}$ and $R_{1}$ determine the outcomes of the diagram (i.e. number of possible solutions), which will be investigated systematically by Yu-Kai through the $R_{1}$ (i.e. $\kappa$ in Yu-Kai's report) versus $\xi_{1}$ diagram (, and the train of thought is same with what is shown below). On the $\mathcal{P}$ versus $\mathcal{V}$ diagram, we also plot the red solid curve `isothermal' as $\mathcal{P}=1/\mathcal{V}$ and the green solid curve `cross' as $\mathcal{P}=\mathcal{V}/|2\mathcal{V}-1|$ (which has a pole at $\mathcal{V}=0.5$) to evaluate the values of $\tau$ and $R_{2}=u_{2}^{2}/a_{2}^{2}$ from the position of $(\mathcal{P}_{*},\ \mathcal{V}_{*})$. The reasoning is as follows.
\begin{align}
R_{2}=\frac{u_{2}^{2}}{a_{2}^{2}}=\frac{u_{1}^{2}}{a_{1}^{2}}\left(\frac{a_{1}}{a_{2}}\right)^{2}\left(\frac{u_{2}}{u_{1}}\right)^{2}=\frac{\mathcal{V}(\mathcal{P}-1)}{\mathcal{P}(1-\mathcal{V})}\ ,\label{e45}
\end{align}
therefore, $R_{2}<1$ is equivalent to $(2\mathcal{V}-1)\mathcal{P}<\mathcal{V}$, which means that the down stream side is in the subsonic region. When properties of the upstream is given, we require that i) $X=u_{1}/u_{2}>1$, i.e $|u_{s}-u_{u}|>|u_{s}-u_{d}|$. We may further by physical considerations require that ii) $\tau^{2}>1$ (i.e. downstream temperature is higher than upstream temperature: $a_{2}>a_{1}$) and iii) $R_{2}< 1$ (i.e. flow of the downstream side is subsonic: $u_{2}/a_{2}<1$, and the shock `crosses' the SCL on the $-v$ versus $x$ plane). These requirements can be shown by formulae~(\ref{e42}) and (\ref{e45}) equivalent to
\begin{align}
\mathrm{i)}&\ \mathcal{V}_{*}<1\, \label{e46}\\
\mathrm{ii)}&\ \mathcal{P}_{*}>\frac{1}{\mathcal{V}_{*}}\ ,\label{e47}\\
\mathrm{iii)}&\ \mathcal{P}_{*}<\frac{\mathcal{V}_{*}}{|2\mathcal{V}_{*}-1|}\ .\label{e48}
\end{align}
The above inequalities illustrate the relation between qualitative features of the shock embodied by self-similar parameters (i.e. $\tau$ and $X$) with the position of $(\mathcal{P}_{*},\ \mathcal{V}_{*})$. 

The third step is to know how quantitative properties of self-similar solutions (i.e. $x_{*}(u)$ and $x_{su}$) are related to the values of $\xi_{1}=-\mathrm{ln}\alpha_{1}$ and $R_{1}$, which tell us the shape of `shock isothermal' curve and possible intersection point(s) on the line named `conservation'. We know that $\alpha\rightarrow A/x^{2}$ as $x\rightarrow+\infty$ and generally $\alpha_{1}$ decreases as $x_{su}$ increases, therefore, $\xi_{1}$ increases as $x_{su}$ increases, and the shape of `shock isothermal' curve varies from Figure~\ref{7} to Figure~\ref{8} then to Figure~\ref{9}. As for $R_{1}$, we have $R_{1}^{1/2}=u_{1}/a_{1}=x_{su}-v_{u}>1$, which increases as $x_{su}$ increases, given $x_{*}(u)$ (see Figure~\ref{1}-\ref{6}).

Now we start to interpret the numerical results in Figure~\ref{1}-\ref{6} by relate them to the exemplar $\mathcal{P}$ versus $\mathcal{V}$ diagrams in Figure~\ref{7}-\ref{9}. As said by Yu-Kai, if $\xi_{1}<-0.807$, there are two poles on the curve `shock isothermal', of which the first locates at $0<\mathcal{V}_{p1}<0.5$, while the second locates at $\mathcal{V}_{p2}>0.5$. Besides, 1) for $\xi_{1}<-1$, we have $\mathcal{V}_{p2}>1$ (see Figure~\ref{7}), while 2) for $-1<\xi_{1}<-0.807$, we have $\mathcal{V}_{p2}<1$ (see Figure~\ref{8}), and the second pole disappears at $\mathcal{V}_{p2}=1$ for $\xi_{1}=-1$, since the null point of the denominator coincides with that of the numerator
on the right hand side of expression~(\ref{e44}) of `shock isothermal' curve. 3) For $\xi_{1}>-0.807$, there is no pole (see Figure~\ref{9}).

When $x_{*}(u)$ is not large, the $x_{su}$s involved are small as well, so the situation that $\xi_{1}<-0.807$ will occur, which is what we found in Figure~\ref{1}-\ref{3} and Figure~\ref{10}-\ref{11}. Given $x_{*}(u)$, as $x_{su}$ increases, $\xi_{1}$ increases, and meanwhile $R_{1}>1$ increases, in which process $\mathcal{V}_{p2}$ decreases and may pass the special value $\mathcal{V}_{p2}=1$. 

1) Before $\mathcal{V}_{p2}$ goes below 1, we are in Figure~\ref{7} in which there could be only one solution that satisfies all the three requirements (\ref{e46})-(\ref{e48})\footnote{It is possible to have solutions which satisfy requirements i) and ii) while violate requirement iii)}, this case corresponds to the numerical results of $x_{su}\sim 0.02-0.05$ with $x_{*}(u)=0.01$ (Figure~\ref{1}), $x_{su}\sim 0.12-0.21$ with $x_{*}(u)=0.1$ (Figure~\ref{2} and \ref{10}), and $x_{su}\sim 0.42-0.48$ with $x_{*}(u)=0.4$ (Figure~\ref{3} and \ref{11})\footnote{I have done numerical trials to verify that indeed here we can find only one solution.}. 

2) After $\mathcal{V}_{p2}$ goes below 1, there are two poles locate in the interval $0<\mathcal{V}<1$, and the curve between these two poles will move upwards as $\xi_{1}$ increases. Finally, when $\xi_{1}$ is close to $-0.807$, this part of `shock isothermal' curve goes to infinity , and the two poles tend to coincide and disappear. So given $x_{*}(u)$, as $x_{su}\nearrow$ (from now on I use `$\nearrow$' to denote increase and `$\searrow$' decrease), $\xi_{1}\nearrow$, and $R_{1}\nearrow$, which is accompanied by the upward movement of `shock isothermal' curve between the two poles. If $R_{1}\nearrow$ fast enough to catch up the up-moving `shock isothermal' curve, we are in Figure~\ref{8} where usually two solutions $\mathcal{V}_{*}(1)<\mathcal{V}_{*}(2)$ can be found, both satisfying requirements i) and ii), i.e. inequalities (\ref{e46}) and (\ref{e47}), while only one of them, i.e. $\mathcal{V}_{*}(1)$, may meet requirement iii) denoted by inequality (\ref{e48}). This corresponds to $x_{su}=0.22$ with $x_{*}(u)=0.1$ (solutions related to $\mathcal{V}_{*}(2)$ and $\mathcal{V}_{*}(1)$ are shown separately in Figure~\ref{2} and \ref{10}), and $x_{su}=0.52$ with $x_{*}(u)=0.4$ (solutions related to $\mathcal{V}_{*}(2)$ and $\mathcal{V}_{*}(1)$ are shown separately in Figure~\ref{3} and \ref{11}). However, if the increase of $R_{1}$ cannot catch up `shock isothermal' curve, there is no solution under requirement ii)~(\ref{e47}), in that only between the two poles, is `shock isothermal' curve above the `isothermal' curve\footnote{But it is still possible to obtain solutions with $\mathcal{P}_{*}<1/\mathcal{V}_{*}$, which could be the case of some solutions of small $x_{su}-x_{*}(u)$ in Figure~\ref{4} and \ref{12}.}. That is why in our numerical trials, we cannot find any solution if $x_{su}$ is too large.

When $x_{*}(u)$ is large and so does $x_{su}$ as well as $\xi_{1}$, we should only consider the intersection of `conservation' line and the `shock isothermal' curve in the range $\mathcal{V}_{p2}<\mathcal{V}<1$ if poles still exist, or $0<\mathcal{V}<1$ if there is no pole (i.e. $\xi_{1}>-0.807$). These two cases in which the part of `shock isothermal' curve concerned is always below `isothermal' curve are similar (, in other words, they switch to each other `smoothly'), so we will stay in Figure~\ref{9} for illustration. 

3) Given $x_{*}(2)$, as $x_{su}\nearrow$, and $\xi_{1}\nearrow$, $R_{1}\nearrow$, and the peak of `isothermal' curve moves towards the upper left corner. If $R_{1}\nearrow$ slowly enough (which is true when $R_{1}$ is close to 1) and $\xi_{1}<0$, so that `conservation' line does not exceed the tangent of `shock isothermal' curve that passes the point $(\mathcal{V},\ \mathcal{P})=(1,\ 1)$, we can obtain two positive solutions $\mathcal{V}_{*}(1)<\mathcal{V}_{*}(2)$ which satisfy requirements i)~(\ref{e46}) and iii)~(\ref{e48}) but do not meet requirement ii)~(\ref{e47})\footnote{It is possible to have some $\mathcal{V}_{*}(2)$ that does not meet requirement iii).}. While $R_{1}$ gets larger, `conservation' line may exceed the tangent, then we have no solution, which is exactly the case that $x_{su}$ is large. Self-similar solutions corresponding to $\mathcal{V}_{*}(2)$ are shown in Figure~\ref{4}-\ref{6}, while their counterparts related to $\mathcal{V}_{*}(1)$ in Figure~\ref{12}-\ref{14}. If $\xi_{1}>0$, $F_{\xi_{1}}(\mathcal{V}=0)=2\xi_{1}+1>1$, so it is possible that $\mathcal{V}_{*}(1)<0$ (when $1<R_{1}<1+2\xi_{1}$), in which case we have only one valid solution $\mathcal{V}_{*}(2)$. 

By now, I have interpreted all the numerical results. The analysis above is generally qualitative and incomplete for understanding all properties of such isothermal shocks and their physical meanings. Yu-Kai will calculate the tangents of different parts of `shock isothermal' curve (that passes (1, 1)) to plot a $R_{1}$ versus $\xi_{1}$ diagram from which we can read information of the solutions given upstream properties or downstream properties. 

\begin{figure}
\centering
\includegraphics[width=1.0\columnwidth]{P-V_3.pdf}
\caption{(Situation C) $\mathcal{P}$ versus $\mathcal{V}$ diagram for $\xi_{1}=-1.2<-1$ and $R_{1}=2.0$ with two poles $0<\mathcal{V}_{p1}<0.5$ and $1<\mathcal{V}_{p2}$. We find one intersection point of the blue solid curve `shock isothermal' and the blue dashed line `conservation' above the red solid curve `isothermal' while below the green solid curve `cross' (see requirements~(\ref{e47}) and (\ref{e48})). This intersection point corresponds to a shock with $a_{2}>a_{1}$, and $R_{2}<1$. It is also possible to find an intersection point with $\mathcal{V}_{*}>1$, which is not shown here and corresponds to the case that downstream properties are known (i.e. $X<1$).}
\label{7}
\end{figure}

\begin{figure}
\centering
\includegraphics[width=1.0\columnwidth]{P-V_2.pdf}
\caption{(Situation B) $\mathcal{P}$ versus $\mathcal{V}$ diagram for $-1<\xi_{1}=-0.9<-0.807$ and $R_{1}=3.0$ with two poles $0<\mathcal{V}_{p1}<0.5$ and $0.5<\mathcal{V}_{p2}<1$. We find two intersection points (i.e. $\mathcal{V}_{*}(1)<\mathcal{V}_{*}(2)$) of the blue solid curve `shock isothermal' and the blue dashed line `conservation' both above the red solid curve `isothermal' (i.e. $a_{2}>a_{1}$). The first one corresponding to $\mathcal{V}_{*}(1)$ is below the green solid curve `cross' (i.e. $R_{2}<1$), while the second one of $\mathcal{V}_{*}(2)$ above the green solid curve `cross' (i.e. $R_{2}>1$). This is the case in which one type of upstream gives one type of subsonic downstream ($\mathcal{V}_{*}(1)$) and another one of supersonic downstream ($\mathrm{V}_{*}(2)$). }
\label{8}
\end{figure}

\begin{figure}
\centering
\includegraphics[width=1.0\columnwidth]{P-V_1.pdf}
\caption{(Situation A) $\mathcal{P}$ versus $\mathcal{V}$ diagram for $\xi_{1}=-0.5>-0.807$ and $R_{1}=1.2$ with no pole. We find two intersection points (i.e. $\mathcal{V}_{*}(1)<\mathcal{V}_{*}(2)$) of the blue solid curve `shock isothermal' and the blue dashed line `conservation' both below the red solid curve `isothermal' while and the green solid curve `cross' (i.e. $a_{2}<a_{1}$, and $R_{2}<1$). The intersection point with $\mathcal{V}_{*}(2)$ is too close to 1 to be seen here. This is the case in which one upstream gives two types of subsonic downstream both violating the restriction that $a_{2}<a_{1}$.}
\label{9}
\end{figure}

\begin{figure}
\centering
\includegraphics[width=1.0\columnwidth]{shock16.pdf}
\caption{(To be compared with Figure~\ref{2}) $-v$ versus $x$ plots of some Shocks under Condition 2 (dynamical) isothermal MHD energy conservation for $\lambda=10^{-10}\approx 0$ and $x_{*}(u)=0.1$. Here we also have both $\tau>1$ and $X>1$. The solutions with $x_{su}\sim 0.12-0.21$ are the same with those in Figure~\ref{2} corresponding to the case illustrated by Figure~\ref{7}. The vacancies in the sequence of $x_{su}$s have been filled with dotted curves, which means that it is because of bad choice of the initial condition to solve Condition 2 transcendental equation set that we found no solution at some $x_{su}$s in Figure~\ref{2}. The solution with $x_{su}=0.22$ corresponds to $\mathcal{V}_{*}(1)$ in Figure~\ref{8}, while its counter part in Figure~\ref{2} $\mathcal{V}_{*}(2)$ in Figure~\ref{8}. Again, if $x_{su}-x_{*}(u)$ is too large, there is no solution.}
\label{10}
\end{figure}

\begin{figure}
\centering
\includegraphics[width=1.0\columnwidth]{shock17.pdf}
\caption{(To be compared with Figure~\ref{3}) $-v$ versus $x$ plots of some Shocks under Condition 2 (dynamical) isothermal MHD energy conservation for $\lambda=10^{-10}\approx 0$ and $x_{*}(u)=0.4$. Here we also have both $\tau>1$ and $X>1$ except at $x_{su}=0.5$. The solutions of $x_{su}\sim 0.42-0.5$ are the same with those in Figure~\ref{3}, and belongs to the case in Figure~\ref{7}. The solution with $x_{su}=0.52$ corresponds to $\mathcal{V}_{*}(1)$ in Figure~\ref{8}, while its counter part in Figure~\ref{3} $\mathcal{V}_{*}(2)$ in Figure~\ref{8}. At $x_{su}=0.5$, we have $X<1$, and this is equivalent to the case that properties of the downstream are given and consistent with the very $\mathcal{V}_{*}>1$ implied by Figure~\ref{7}.}
\label{11}
\end{figure}

\begin{figure}
\centering
\includegraphics[width=1.0\columnwidth]{shock13.pdf}
\caption{(To be compared with Figure~\ref{4}) $-v$ versus $x$ plots of some Shocks under Condition 2 (dynamical) isothermal MHD energy conservation for $\lambda=10^{-10}\approx 0$ and $x_{*}(u)=0.8$. Here $X>1$, however, $\tau<1$. These solutions correspond to $\mathcal{V}_{*}(1)$ in Figure~\ref{9}, while their counterparts in Figure~\ref{4} $\mathcal{V}_{*}(2)$ in Figure~\ref{9}. Please ignore those meaningless curves after hitting the SCL with instability.}
\label{12}
\end{figure}

\begin{figure}
\centering
\includegraphics[width=1.0\columnwidth]{shock14.pdf}
\caption{(To be compared with Figure~\ref{5}) $-v$ versus $x$ plots of some Shocks under Condition 2 (dynamical) isothermal MHD energy conservation for $\lambda=10^{-10}\approx 0$ and $x_{*}(u)=1.2$. Here $X>1$, however, $\tau<1$. These solutions correspond to $\mathcal{V}_{*}(1)$ in Figure~\ref{9}, while their counterparts in Figure~\ref{5} $\mathcal{V}_{*}(2)$ in Figure~\ref{9}. Please ignore those meaningless curves after hitting the SCL with instability.}
\label{13}
\end{figure}

\begin{figure}
\centering
\includegraphics[width=1.0\columnwidth]{shock15.pdf}
\caption{(To be compared with Figure~\ref{6}) $-v$ versus $x$ plots of some Shocks under Condition 2 (dynamical) isothermal MHD energy conservation for $\lambda=10^{-10}\approx 0$ and $x_{*}(u)=2.0$. Here $X>1$, however, $\tau<1$. These solutions correspond to $\mathcal{V}_{*}(1)$ in Figure~\ref{9}, while their counterparts in Figure~\ref{6} $\mathcal{V}_{*}(2)$ in Figure~\ref{9}. Please ignore those meaningless curves after hitting the SCL with instability.}
\label{14}
\end{figure}

\bibliographystyle{mnras}
\bibliography{ref} 

\label{lastpage}
\end{document}
